\documentclass[10pt,a4paper]{article}
\usepackage[utf8]{inputenc}
\usepackage[french]{babel}
\usepackage[T1]{fontenc}
\usepackage{graphicx}

\usepackage{fancyhdr}
\usepackage{vmargin}

\setlength{\parindent}{0cm}
\setlength{\parskip}{1ex plus 0.5ex minus 0.2ex}
\newcommand{\hsp}{\hspace{20pt}}
\newcommand{\HRule}{\rule{\linewidth}{0.5mm}}

\begin{document}
	
	\pagestyle{fancy}
	\fancyhf{}
	\rhead{MAAZOUZ Mehdi, LECOCQ Alexis}
	\lhead{Dumbo interpreter}
	\cfoot{\thepage}
	
	\begin{titlepage}
		\begin{sffamily}
			\begin{center}
				% Upper part of the page. The '~' is needed because \\
				% only works if a paragraph has started.
				\includegraphics[scale=0.25]{images/elephant.png}~\\[1.5cm]
				
				% Title
				\HRule \\[0.5cm]
				{ \huge \bfseries Dumbo interpreter\\[0.4cm] }
				\HRule \\[1.5cm]
				
				\Large{Rapport de projet de compilation}\\[2cm]
				
				\Large{Annee Academique 2016-2017}\\[2cm]
				
				% Author and supervisor
				\begin{minipage}{0.4\textwidth}
					\begin{flushleft} \large
						\emph{\textbf{Auteurs :}}\\
						MAAZOUZ Mehdi\\
						LECOCQ Alexis
					\end{flushleft}
				\end{minipage}
				\begin{minipage}{0.4\textwidth}
					\begin{flushright} \large
						\emph{\textbf{Directeurs :}}\\
						BRUYÈRE Véronique\\
						DECAN Alexandre\\
					\end{flushright}
				\end{minipage}
				
				\vfill
				
				% Bottom of the page
				{\large \today}
				
			\end{center}
		\end{sffamily}
	\end{titlepage}
	
	\newpage
	\tableofcontents
	\newpage
	\section{Introduction}
	Dans le cadre du cours de compilation, nous devons réaliser un projet afin de mettre en pratique la théorie vue au cours.
	Le projet doit être écrit en Python 3 et utiliser la librairie ply.
	
	L'objectif du projet est de réaliser un moteur de template à l'aide d'un langage créé pour l'occasion : le dumbo. Nous en décrirons la grammaire dans un prochain chapitre.
	
	Un moteur de template est principalement utilisé pour séparer les données de la manière de les représenter. Notre script dumbo\_interpreter.py doit donc prendre trois arguments :
	\begin{itemize}
		\item data\_file : fichier dumbo contenant les données ;
		\item template\_file : fichier dumbo contenant la présentation des données ;
		\item output\_file : fichier de sortie contenant le fichier template dans lequel les données ont été insérées.
	\end{itemize}
	
	\section{Grammaire du Dumbo}
	Voici la grammaire du langage :\\
	\begin{tabular}{|l l l|}
		\hline
		
		<program> & $\longrightarrow$ & <subprogram> | <subprogram> <program>\\
		<program> & $\longrightarrow$ & <subprogram> <program>\\
		
		\hline
	\end{tabular}
	
	Nous avons essayé de respecter au mieux la grammaire donnée, initialement dans l'énoncé du projet.
	Cependant, nous avons procédé à quelques ajouts et modifications afin d'atteindre les objectis demandés dans 
	l'énoncé.\\
	Pour les opérateurs arithmétiques, nous avons pris la peine de définir dans la grammaire les variables ``term'' ainsi que ``factor'' afin d'éviter
	le cas où la grammaire serait ambiguë.\\
	Concernant l'ajout des booleans, nous nous sommes inspirés du cours. Nous commençons avec la variable boolop qui dérive sur une autre variable
	boolop, un opérateur ``AND'' ou ``OR'' ainsi que la variable bool. Cette dernière correpond au lexème boolean ou à la variable ``comparison''.
	Qui est le résultat de deux variables ``intop'' muni d'un lexème comparator.
	\subsection{Le ``For''}
	\subsection{Le ``If''}
	\section{Problèmes rencontrés}
	\section{Conclusion}
\end{document}
