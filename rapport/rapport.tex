\documentclass[a4paper,10pt]{article}
\usepackage[utf8]{inputenc}
\usepackage{listings}

%opening
\title{Rapport du projet de compilation}
\author{Maazouz Mehdi, Lecocq Alexis}

\begin{document}

\maketitle
\textbf{Annee Academique 2016-2017}\\
\tableofcontents
\newpage

\section{Introduction}
Dans le cadre du cours de Compilation donné en 3 ème Bachelier, nous avons dû réaliser un programme
capable de générer un fichier contenant du texte en fonction de certaines données reçues en paramètre.
Pour ce faire, le langage de programmation utilisé est le Python.
\section{Grammaire}
Nous avons essayé de respecter au mieux la grammaire donnée, initialement dans l'énoncé du projet.
Cependant, nous avons procédé à quelques ajouts et modifications afin d'atteindre les objectis demandés dans 
l'énoncé.\\
Pour les opérateurs arithmétiques, nous avons pris la peine de définir dans la grammaire les variables ``term'' ainsi que ``factor'' afin d'éviter
le cas où la grammaire serait ambiguë.\\
\begin{lstlisting}
def p_intop_plus_minus(p):
  '''intop : intop PLUS term
           | intop MINUS term'''
  p[0] = (p[2], infos(p), p[1] , p[3])

def p_intop_term(p):
  '''intop : term'''
  p[0] = p[1]

def p_term_times_divide(p):
  '''term : term TIMES factor
          | term DIVIDE factor'''
  p[0] = (p[2], infos(p), p[1], p[3] )

def p_term_factor(p):
  '''term : factor'''
  p[0] = p[1]

def p_factor_num(p):
  '''factor : INTEGER'''
  p[0] = p[1]

def p_factor_variable(p):
  '''factor : variable'''
  p[0] = p[1]

def p_variable(p) :
  '''variable : VARIABLE'''
  p[0] = ('variable', infos(p), p[1])

\end{lstlisting}
\newpage
Concernant l'ajout des booleans, nous nous sommes inspirés du cours. Nous commençons avec la variable boolop qui dérive sur une autre variable
boolop, un opérateur ``AND'' ou ``OR'' ainsi que la variable bool. Cette dernière correpond au lexème boolean ou à la variable ``comparison''.
Qui est le résultat de deux variables ``intop'' muni d'un lexème comparator.

\begin{lstlisting}
def p_boolop_boolop_bool(p) :
  '''boolop : boolop AND bool
            | boolop OR bool'''
  p[0] = (p[2], infos(p), p[1], p[3])

def p_boolop_bool(p) :
  '''boolop : bool'''
  p[0] = p[1]

def p_bool_boolean(p) :
  '''bool : BOOLEAN'''
  p[0] = p[1]

def p_bool_comparison(p) :
  '''bool : comparison'''
  p[0] = p[1]

def p_comparison(p) :
  '''comparison : intop COMPARATOR intop'''
  p[0] = (p[2], infos(p), p[1], p[3])
\end{lstlisting}

\subsection{Le ``For''}
\subsection{Le ``If''}
\section{Problèmes rencontrés}
\section{Conclusion}


\end{document}
